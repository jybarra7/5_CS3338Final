\section{System Overview}

The Lunar Rocks! system is a containerized web application designed to support task data submission workflows. 
In Snapshot 1, the system focuses on backend API functionality, container orchestration, and a clearly defined 
request flow.

The application uses a Node.js backend running inside Docker, exposing REST endpoints that allow clients to 
retrieve available tasks and submit task-related data.

\section{System Components}

\subsection{Frontend}
In Snapshot 1, a dedicated frontend is not yet implemented. 
Requests are validated using browser based tools such as curl or a web browser. 
Future snapshots will introduce a user facing interface.

\subsection{Backend}
The backend is implemented using Node.js with the Express framework. 
It provides RESTful endpoints for:
\begin{itemize}
  \item Health checks
  \item Task retrieval
  \item Submission of task data
\end{itemize}

Submissions are stored in memory for Snapshot 1 and will later be migrated to persistent storage.

\subsection{Data Storage}
No external database is used in Snapshot 1. 
All submissions are stored in an inmemory data structure within the backend service.

\section{Workflow Description}

Figure~\ref{fig:workflow} illustrates the system workflow. 
A user initiates a request to the backend API through the exposed service port. 
Docker routes the request to the Node.js backend container, where Express processes the request and returns a JSON 
response.

Task data submissions follow the same flow and are validated before being recorded in memory.

\section{Docker and Deployment}

The system is deployed using Docker Compose. 
A single backend service is defined, exposing port 3000 and running the Node.js application inside a container.

Docker ensures environment consistency and simplifies deployment across systems.

\section{Dependencies and Libraries}

The primary technologies used include:
\begin{itemize}
  \item Node.js
  \item Express.js
  \item Docker
  \item Docker Compose
\end{itemize}

These tools were selected to ensure portability, simplicity, and scalability.

\section{Future Enhancements}

Future snapshots will introduce:
\begin{itemize}
  \item Persistent database storage
  \item A frontend user interface
  \item Automated testing and reporting
\end{itemize}

