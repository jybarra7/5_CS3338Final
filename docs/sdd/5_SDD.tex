\documentclass[12pt]{article}
\usepackage[utf8]{inputenc}
\usepackage{geometry}
\usepackage{titlesec}
\usepackage{hyperref}
\usepackage{booktabs}
\usepackage{tabularx}
\usepackage{graphicx}
\usepackage{float}

% Formatting setup
\geometry{a4paper, margin=1in}
\setlength{\parskip}{1em}
\setlength{\parindent}{0pt}

% Title Page Information
\title{\textbf{Software Design Document (SDD)}\\ \Large Lunar Rocks!}
\author{Group 5}
\date{\today}

\begin{document}

% --- COVER PAGE ---
\maketitle
\thispagestyle{empty}
\vspace{5cm}
\begin{center}
    \textbf{Version 1.0} \\
    CS 3338 - Snapshot 1
\end{center}
\newpage

% --- TABLE OF CONTENTS ---
\tableofcontents
\newpage

% --- VERSION DESCRIPTION ---
\section*{Version Description}
\addcontentsline{toc}{section}{Version Description}

\begin{table}[h]
    \centering
    \begin{tabularx}{\textwidth}{|l|l|X|l|}
        \hline
        \textbf{Version} & \textbf{Date} & \textbf{Description} & \textbf{Author} \\
        \hline
        1.0 & \today & Initial Design for Snapshot 1. Architecture definition and DB Schema V1. & Group 5 \\
        \hline
    \end{tabularx}
\end{table}

\newpage

% --- SECTION 1: INTRODUCTION ---
\section{Introduction}

\subsection{Purpose of the Document}
The purpose of this Software Design Document (SDD) is to provide a comprehensive architectural overview of the "Lunar Rocks!" system. It details the system architecture, data flow, user interface design, and database schema to guide the development process for Snapshot 1.

\subsection{Intended Audience}
This document is designed for the internal development team (Backend and Frontend engineers) and the Faculty Advisor to understand the technical implementation details.

\subsection{Overview of the System}
"Lunar Rocks!" is a full-stack web application. It transitions from a previous legacy Node.js/Prisma stack to a more robust Java/PostgreSQL architecture to reduce costs and improve performance. The system facilitates the interaction between Citizen Scientists (users) and the lunar image database provided by NASA JPL.

% --- SECTION 2: SYSTEM ARCHITECTURE ---
\section{System Architecture}

\subsection{Workflow of the System}
The system follows a standard client-server architecture:
\begin{enumerate}
    \item \textbf{Client-Side (Frontend):} Built with React and Vite. It handles user interactions, image rendering (via Konva.js for drawing tools), and sends HTTP requests to the backend.
    \item \textbf{Server-Side (Backend):} A Java Web Service running on Apache Tomcat. It uses Java Jakarta EE and Servlets to process API requests.
    \item \textbf{Database:} PostgreSQL with PostGIS extensions. The Java backend communicates with the database via direct SQL queries (JDBC).
\end{enumerate}

\begin{figure}[H]
    \centering
    \includegraphics[width=0.95\textwidth]{assets/3338Diagram.png}
    \caption{High-level workflow of the Lunar Rocks! system}
    \label{fig:lunar-workflow}
\end{figure}

\subsection{Data Flow Diagrams}

% ---- Level 0 DFD placeholder ----
\begin{figure}[H]
    \centering
    % ensure the file Level0DFD.png (or .jpg) is in the same folder
    \includegraphics[width=0.75\textwidth]{Level0DFD}
    \caption{Level 0 Data Flow Diagram for Lunar Rocks!}
\end{figure}

% ---- Level 1 DFD placeholder ----
\begin{figure}[H]
    \centering
    % ensure the file Level1DFD.png (or .jpg) is in the same folder
    \includegraphics[width=\textwidth]{Level1DFD}
    \caption{Level 1 Data Flow Diagram for Lunar Rocks!}
\end{figure}

\subsection{Component Breakdown}
\begin{itemize}
    \item \textbf{Frontend:}
    \begin{itemize}
        \item \texttt{React.js}: Component-based UI structure.
        \item \texttt{TailwindCSS}: Styling and responsiveness.
        \item \texttt{Konva}: Canvas manipulation for Sizing tasks.
    \end{itemize}
    \item \textbf{Backend:}
    \begin{itemize}
        \item \texttt{Java}: Core logic language.
        \item \texttt{Tomcat}: Web server environment.
        \item \texttt{Jakarta EE}: Enterprise features for web services.
    \end{itemize}
\end{itemize}

% --- SECTION 3: USER INTERFACE ---
\section{User Interface}

\subsection{How to Use the System}
\begin{enumerate}
    \item \textbf{Login/Registration:} Users arrive at the landing page and authenticate via the "Sign In" portal (supporting email or social login).
    \item \textbf{Dashboard:} Upon login, users are presented with the "Explore" page where they select a task (Scout, Size, or Classify).
    \item \textbf{Task Execution:}
    \begin{itemize}
        \item \textit{Scouting:} The user views a grid of images and selects a range representing the rock count.
        \item \textit{Sizing:} The user utilizes mouse/touch input to draw shapes around rocks. Tools include Undo, Redo, and Clear.
        \item \textit{Classification:} The user selects a highlighted rock and assigns a shape attribute (e.g., Rounded).
    \end{itemize}
\end{enumerate}

\subsection{Screen Frameworks (Screenshots)}

% Home / Landing page
\begin{figure}[H]
    \centering
    \includegraphics[width=0.95\textwidth]{8.2ScreenFrameworks}
    \caption{Home page layout}
\end{figure}

% About Us page
\begin{figure}[H]
    \centering
    \includegraphics[width=0.95\textwidth]{8.2.1ScreenFrameworks}
    \caption{About Us page}
\end{figure}

% FAQ + Contact
\begin{figure}[H]
    \centering
    \includegraphics[width=0.95\textwidth]{8.2.2ScreenFrameworks}
    \caption{FAQ and Contact pages}
\end{figure}

% Explore + Admin
\begin{figure}[H]
    \centering
    \includegraphics[width=0.95\textwidth]{8.2.3ScreenFrameworks}
    \caption{Explore Tasks and Admin Dashboard}
\end{figure}

% Scouting / Sizing / Classification tasks
\begin{figure}[H]
    \centering
    \includegraphics[width=0.95\textwidth]{8.2.4ScreenFrameworks}
    \caption{Example task screens: Scouting, Sizing, and Classification}
\end{figure}

\subsection{Database Design and Explanation}
The database utilizes PostgreSQL. Key tables include:
\begin{itemize}
    \item \textbf{Users:} Stores \texttt{id}, \texttt{username}, \texttt{password} (hashed), and \texttt{role}.
    \item \textbf{Images:} Stores metadata about lunar images including \texttt{image\_url}, \texttt{resolution}, and \texttt{capture\_date}.
    \item \textbf{UserGeometry:} Stores the spatial data drawn by users during the sizing task (PostGIS geometry types).
    \item \textbf{Classifications:} Links specific rocks to user-selected shape categories.
\end{itemize}

% ---- Database diagram placeholder ----
\begin{figure}[H]
    \centering
    \includegraphics[width=\textwidth]{DatabaseDesign}
    \caption{Proposed Database Schema for Lunar Rocks!}
\end{figure}

% --- SECTION 4: GLOSSARY ---
\section{Glossary}

\begin{table}[H]
    \centering
    \begin{tabularx}{\textwidth}{|l|X|}
        \hline
        \textbf{Acronym} & \textbf{Definition} \\
        \hline
        UI & User Interface \\
        \hline
        API & Application Programming Interface \\
        \hline
        DB & Database \\
        \hline
        SQL & Structured Query Language \\
        \hline
        PostGIS & Spatial Database Extender for PostgreSQL \\
        \hline
        ORM & Object-Relational Mapping \\
        \hline
    \end{tabularx}
\end{table}

% --- SECTION 5: REFERENCES ---
\section{References}

\begin{enumerate}
    \item Previous Team SRS/SDD: ``Viper Rocks! Software Requirements and Design.'' 
    \item JPL Design System: \url{https://www.jpl.nasa.gov/brand}
    \item React Documentation: \url{https://react.dev/}
    \item PostGIS Documentation: \url{https://postgis.net/}
\end{enumerate}

\end{document}
