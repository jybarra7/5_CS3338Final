
\documentclass[11pt]{article}

\usepackage[margin=1in]{geometry}
\usepackage{hyperref}
\usepackage{graphicx}
\usepackage{longtable}
\usepackage{array}
\usepackage{xcolor}

\hypersetup{
  colorlinks=true,
  linkcolor=black,
  urlcolor=blue,
  citecolor=black
}

\setlength{\parskip}{0.5em}
\setlength{\parindent}{0pt}

\begin{document}

% ---------------- Cover Page ----------------
\begin{titlepage}
    \centering
    \vspace*{1.5cm}

    {\LARGE \textbf{Software Design Document (SDD)}\par}
    \vspace{0.5cm}
    {\Large \textbf{Lunar Rocks!}\par}

    \vspace{1.5cm}

    {\large Version 1.1\par}
    \vspace{0.25cm}
    {\large CS 3338 -- Lab Assignment 14\par}

    \vspace{1.5cm}

    {\large \textbf{Team Name:} Team 5\par}

    \vfill

    {\large \textbf{Date:} \today\par}

\end{titlepage}

% ---------------- Table of Contents ----------------
\tableofcontents
\newpage

% ---------------- Version Description ----------------
\section*{Version Description}
\addcontentsline{toc}{section}{Version Description}

\begin{longtable}{|>{\raggedright\arraybackslash}p{2.5cm}|>{\raggedright\arraybackslash}p{10cm}|>{\raggedright\arraybackslash}p{2.5cm}|}
\hline
\textbf{Version} & \textbf{Description} & \textbf{Date} \\
\hline
1.0 & Initial SDD for Snapshot 1. Includes system overview, architecture, UI description, database summary, glossary, and references. & \today \\
\hline
1.1 &
Snapshot 2: Design expanded to include authenticated user workflows.
Introduces authentication components, dashboard design, updated system
workflow, and supporting dependencies. & \today \\
\hline
1.2 &
Snapshot 3: Design extended to support full task execution.
Defines task interaction components, validation strategy, submission flow,
and data persistence design for scouting, sizing, and classification tasks. &
\today \\
\hline
\end{longtable}

\newpage

% ---------------- Introduction ----------------
\section{Introduction}

\subsection{Purpose of the Document}
This Software Design Document (SDD) describes the high-level and component-level
design of the \textit{Lunar Rocks!} web application. The goal is to provide a clear blueprint
for implementation, maintenance, and future expansion of the system.

\subsection{Intended Audience}
This document is intended for:
\begin{itemize}
    \item Project stakeholders and sponsors.
    \item Faculty advisors and course evaluators.
    \item Future developers who may inherit or extend the codebase.
    \item QA and documentation team members.
\end{itemize}

\subsection{Overview of the System}
\textit{Lunar Rocks!} is a citizen science web platform that allows users to
\textbf{scout}, \textbf{size}, and \textbf{classify} lunar rocks using image-based tasks.
The platform is designed to streamline lunar rock analysis and help build
a unified dataset that supports lunar geology research and mission planning.
The project evolved from the earlier \textit{VIPER Rocks!} initiative and continues as
\textit{Lunar Rocks!} even as the VIPER mission direction changed.\footnote{This historical
context is included for completeness and can be shortened for a strict scope.}

% ---------------- System Architecture ----------------
\section{System Architecture}

\subsection{Workflow of the System}
With authentication enabled, the system workflow is updated as follows:
\begin{enumerate}
    \item User logs in and accesses the dashboard.
    \item User selects a task type.
    \item System loads task-specific interface and data.
    \item User interacts with the task interface.
    \item System validates task input.
    \item User submits task results.
    \item Backend persists results in the database.
\end{enumerate}

\subsection{Component Breakdown}
The system follows a full-stack web architecture with the following major components:

\subsubsection*{Frontend Responsibilities}
The frontend authentication components are responsible for:
\begin{itemize}
    \item Providing registration and login forms
    \item Performing basic input validation (required fields, formatting)
    \item Displaying authentication success or error messages
    \item Redirecting authenticated users to the dashboard
\end{itemize}

\subsubsection*{Backend Responsibilities}
The backend authentication components are responsible for:
\begin{itemize}
    \item Handling registration, login, and logout requests
    \item Securely hashing and storing user passwords
    \item Managing authenticated user sessions
    \item Enforcing access control for task-related endpoints
\end{itemize}

\subsubsection*{Database Layer}
\begin{itemize}
    \item Uses \textbf{PostgreSQL}, with spatial capabilities supported by \textbf{PostGIS} as needed.
    \item Stores user accounts, images, task definitions, and task results.
\end{itemize}

\subsection{Data Flow Summary}
The conceptual data flow includes:
\begin{itemize}
    \item \textbf{User} $\rightarrow$ \textbf{Frontend UI} for task interactions.
    \item \textbf{Frontend} $\rightarrow$ \textbf{Backend APIs} via HTTP requests.
    \item \textbf{Backend} $\rightarrow$ \textbf{Database} for persistent storage and retrieval.
    \item \textbf{Database} $\rightarrow$ \textbf{Backend} $\rightarrow$ \textbf{Frontend} for dashboard and progress views.
\end{itemize}

% ---------------- User Interface ----------------
\section{User Interface}
\subsection{Dashboard Component Design}
The Dashboard serves as the central navigation hub for authenticated users.
It provides:
\begin{itemize}
    \item Entry points for Scouting, Sizing, and Classification tasks
    \item Navigation links to informational pages
    \item User session controls (e.g., log out)
\end{itemize}

\subsection{Design Dependencies Introduced}
Snapshot 2 introduces the following design dependencies:
\begin{itemize}
    \item Password hashing utilities (e.g., bcrypt or equivalent)
    \item Authentication/session management mechanisms
\end{itemize}

\subsection{Testing and Documentation Artifacts}
Snapshot 2 design updates are supported by:
\begin{itemize}
    \item A Snapshot 2 Objective document outlining the authentication feature
    \item TestRail test cases covering login, logout, and dashboard routing
    \item Updates to the user manual and design documentation reflecting the new workflow
\end{itemize}

\subsection{Task Interfaces}

\subsubsection{Scouting Interface}
The scouting page presents a lunar image and asks the user to estimate how many
rocks appear in the image or region. This step supports partitioning and helps prepare
datasets for later tasks.

\subsubsection{Sizing Interface}
The sizing task provides drawing and measurement tools so users can select rocks and
estimate dimensions. Features may include drag-and-drop rock selection, precision drawing,
undo/redo, and save progress.

\subsubsection{Classification Interface}
The classification task highlights a rock outline and asks the user to choose the shape category
that best matches the rock. The UI should support ambiguity handling (e.g., skip option)
to avoid forcing uncertain classifications.

\subsection{Validation and Submission Design}
Validation occurs at multiple stages:
\begin{itemize}
    \item Client-side validation ensures required inputs are provided.
    \item Server-side validation enforces data integrity and format.
\end{itemize}

\subsection{Supporting Dependencies}
The task execution design relies on:
\begin{itemize}
    \item Interactive canvas or image-handling libraries
    \item Validation utilities on both client and server
\end{itemize}

% ---------------- Database Design ----------------
\section{Database Design}

\subsection{Overview}
The database is responsible for storing:
\begin{itemize}
    \item User authentication and profile data.
    \item Task definitions and task image references.
    \item User submissions for scouting, sizing, and classification.
    \item Administrative metadata (e.g., aggregation status, audit trails).
\end{itemize}

\subsection{Conceptual Entities}
Based on the project’s documented schema direction, core entities likely include:
\begin{itemize}
    \item \textbf{User/Account}
    \item \textbf{Image}
    \item \textbf{Task}
    \item \textbf{TaskType}
    \item \textbf{TaskData} (user submissions)
\end{itemize}

\subsection{Design Rationale}
A relational approach supports strong data integrity and indexing for large datasets.
The backend can issue SQL queries directly (via the web service layer) to optimize
performance and reduce reliance on paid third-party tools.

% ---------------- Glossary ----------------
\section{Glossary}

\begin{longtable}{|p{3cm}|p{12cm}|}
\hline
\textbf{Term} & \textbf{Definition} \\
\hline
API & Application Programming Interface; defines how the frontend communicates with backend services. \\
\hline
UI & User Interface; the visual components users interact with. \\
\hline
DB & Database; persistent storage of system data. \\
\hline
ORM & Object Relational Mapper; a tool that maps objects in code to database tables. \\
\hline
PostGIS & A spatial database extension for PostgreSQL. \\
\hline
Tomcat & A Java application server used to deploy web services. \\
\hline
\end{longtable}

% ---------------- References ----------------
\section{References}
\begin{itemize}
    \item Internal project documentation and presentations.
    \item PostgreSQL and PostGIS official documentation.
    \item React and Vite official documentation.
\end{itemize}

\end{document}
